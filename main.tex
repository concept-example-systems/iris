\documentclass{IOS-Book-Article}

\usepackage{mathptmx}
\usepackage{url}
\usepackage{amsthm}
\theoremstyle{definition}
\newtheorem{example}{Example}[section]
\newtheorem{definition}{Definition}[section]

%\usepackage{times}
%\normalfont
%\usepackage[T1]{fontenc}
%\usepackage[mtplusscr,mtbold]{mathtime}
%
\begin{document}
\begin{frontmatter}              % The preamble begins here.

%\pretitle{Pretitle}
\title{Giving Examples Instead of Answering Questions: Introducing Legal Concept-Example Systems}
\runningtitle{Legal Concept-Example Systems}
%\subtitle{Subtitle}

\author[A]{\fnms{Tomer} \snm{Libal}
\thanks{Supported by the Luxembourg National Research Fund under grant C22/IS/17228828/ExAlLe}},
\author[B]{\fnms{Aleksander} \snm{Smywiński-Pohl}
\thanks{Supported by the Polish National Centre for Research
and Development – Pollux Program under Grant WM/POLLUX11/5/2023
titled ,,Examples based AI Legal Guidance''. We gratefully acknowledge Polish high-performance computing infrastructure PLGrid
(HPC Centers: ACK Cyfronet AGH) for providing computer facilities and support within computational grant no. PLG/2023/016304.}}

\runningauthor{Libal and Smywiński-Pohl}
\address[A]{Department of Computer Science, Luxembourg University, Luxembourg}
\address[B]{Computer Science Institute, AGH University of Krakow, Poland}

\begin{abstract}
Question-Answering Systems (QASs) have seen a big development in recent years and various attempts have been made to extend them to the legal domain. Nevertheless, the needs and methodology of legal research relies often less on getting answers and more on finding positive and negative examples for certain legal concepts. In this paper, we introduce a sub-category within QASs that focuses on such legal tasks, motivate its usefulness, design a methodology and demonstrate its applicability on a small example.
\end{abstract}

\begin{keyword}
Question-Answering Systems \sep Legal Research \sep
Automatic Questions Generation \sep GDPR
\end{keyword}
\end{frontmatter}

\thispagestyle{empty}
\pagestyle{empty}

\pagenumbering{arabic}
\section{Introduction}

Question Answering Systems (QASs) are prominent in making knowledge accessible. Systems such as, for example, in health care (for a survey, see \cite{budler2023review}), have seen rising popularity. Similar attempts have also been made in the legal domain. The COLIEE competition \footnote{\url{https://sites.ualberta.ca/~rabelo/COLIEE2023}} had until recently the Legal Question-Answering challenge, which in the last competitions has been subsumed by Task 3 (Statute Law Retrieval Task) and Task 4 (Legal Textual Entailment Data Corpus). 

Martinez-Gil \cite{martinez2023survey} gives a comprehensive survey of the different approaches and tools and concludes that no solution has managed to provide high accuracy, interpretability, and performance. The reason being that accuracy and inerpretability are mainly obtained by investing considerable human effort, for example those based on ontologies (e.g. \cite{fawei2018methodology}) and linked-data (e.g. \cite{filtz2021linked}).

An important challenge for legal professionals is legal research \cite{sanderson2021practical} due to the ever changing statute and case law. As a consequence of their volume and dynamics, a legal QAS must meet the performance property. Legal QASs in general should also meet the accuracy and interpretability properties. 

The problem addressed in this paper is how the three properties can be obtained within one system, for the purpose of enhancing the legal research process. In order to approach this problem, we first investigate (Sec. \ref{sec:lces}) the needs of legal professionals and characterize the form of Legal QASs that meets their legal research needs. Our main finding is that while legal QASs are useful for professionals, their current methodology involves less looking for answers to specific questions and more looking for positive and negative examples to specific legal concepts within case law. We therefore introduce a new approach to QASs, called Concept-Example Systems (CESs). 

We then follow by surveying the current state-of-the-art in legal QASs with regards to this analysis (Sec. \ref{sec:sota}), where we also argue why our system is better suited for the legal research. The SotA section is followed by an introduction to our methodology (Sec. \ref{sec:method}), as well as a prototype implementation and experiment (Sec. \ref{sec:experiments}). We conclude in the last section.

\section{Legal Concept-Example Systems (LCESs)}
\label{sec:lces}

\subsection{The legal research process}

Before we can present Concept-Example Systems, we would like to understand why they might be useful. We therefore devote the first section to an exposition of the legal research process. For that purpose, we follow the description by Sanderson and Kelly in their ``A practical Guide to Legal Research'' \cite{sanderson2021practical} and take advantage of their running example. 
%It should be noted that while the example takes place in Australia and is based on Australian law, the process in other legal systems does not differ much when considering legal research.
The example is about two Pakistani men, Aban and Salim, who have arrived in Australia with visitor visas and sought refugee status before their visas have expired. They claim they would be prosecuted for being homosexual if they are forced to return to Pakistan and risk being beaten or killed.

Our goal is to build their case for an asylum status in Australia and for that, we need to conduct legal research. 
In their guide, Sanderson and Kelly define the following steps of legal research:

\begin{definition}[Steps of Legal Research \cite{sanderson2021practical}]
The main steps of legal research are:
\begin{enumerate}
    \item identification of the relevant facts;
    \item identification of the legal issues;
    \item identification and interpretation of the rules that govern the legal issues;
    \item application of the rules to the facts; and
    \item conclusion/s.
\end{enumerate}
\end{definition}

Our focus in this paper is step 3. We therefore briefly conclude the findings of the first two steps and ignore steps 4 and 5.

The first step concludes with the following facts - Refugee; Persecution; Homosexual. 
For identifying the legal issues (Step 2), we proceed to identify key legal questions with regards to the facts.

\begin{itemize}
\item Are Aban and Salim refugees? What does the term refugee mean?
\item Can Aban and Salim be persecuted because they are homosexual?
\end{itemize}

Steps 3 guides us in the legal research, which is the main topic of this paper. The key questions to answer in this step are:

\begin{enumerate}
\item What is the relevant legislation? 
\item What cases, if any, have considered similar issues? 
\item Are there relevant international treaties or conventions?
\end{enumerate}

Our paper focuses on legal research about cases, i.e. item 2 above. We therefore omit items 1 and 3. We nevertheless, follow their reasoning and conclude that the relevant legislation is Migration Act 1958 (Cth)\footnote{\url{http://www5.austlii.edu.au/au/legis/cth/consol_act/ma1958118/index.html}}, which references the United Nations Convention Relating to the Status of Refugees\footnote{\url{https://www.unhcr.org/media/convention-and-protocol-relating-status-refugees}}.

From the legislation, we have questions such as:
\begin{itemize}
\item what is a well-founded fear of persecution?
\item is homosexuality a ‘social group’ for the purposes of the Migration Act 1958?
\end{itemize}

The guide states that ``Case law should assist us to determine the meaning of ‘well-founded fear’ and ‘social group'' \cite{sanderson2021practical} and suggests three methods for doing that: (1) by using search terms, (2) by starting from a known similar case, (3) by looking on judicial analysis of a certain legislation. In this paper we focus on the first option.
Modern case law search is mainly done by using special tools, such as the ones offered by LexisNexis and Thomson Reuters Westlaw, or by publicly available indexes, such as EUR-Lex\footnote{\url{http://eur-lex.europa.eu}}.

The main search process includes using search terms and filters in order to locate relevant judgements. If we input in an Australian case law search tool the keywords ``well-founded fear'', together with keywords such as ``homosexuality'' and ``prosecution'', many relevant cases are returned, such as ``S395 v Minister for Immigration and Multicultural Affairs''. 

Once a set of relevant cases was identified, the legal professional needs to read the cases, as summaries or digests are not enough.
A reading of the above mentioned case gives the following paragraph: ``the Tribunal had misdirected itself on the issue of discretion and had not properly considered the appellants’ claims that they had a real and well-founded fear of persecution if they returned to Bangladesh. The real question was whether the harm that would be suffered was such that, {\bf by reason of its intensity or duration, the person cannot reasonably be expected to tolerate it}''.

This paragraph and its concluding remark gives a positive example for a high court decision whether the keyword ``well-founded fear'' holds under the context ``homosexuality'' and ``prosecution'' and under the Australian ``Migration Act 1958''. By using this example, the legal professional can return to her case at hand at determine if the example is similar or not. Clearly, further examples and especially a mix of positive and negative ones would provide the most useful help in determining that.

We argue in this paper that one of the most useful questions for a legal QAS would be to show positive and negative examples of a concept under a certain context. When considering a general purpose legal QAS, it would need to understand the meaning of the question and would not be optimized for finding examples, thus increasing the chance for error, as well as hampering interpretability. It should be noted though, that there are also many cases when the most useful question to ask is not of the above form.

We can now describe a new type of QASs which is tailored for the task above. A methodology for creating such a system, as well as a simple experiment, are described in sections \ref{sec:method} and \ref{sec:experiments}.

\subsection{Concept-Example Systems (CESs)}

A Concept-Example System (CES) differs from a QAS in two aspects:

\begin{itemize}
 \item The starting point for generating a question-answer pair is a legal concept and not a question.
 \item We do not look for specific answers but for as many extracts from court judgements which resolve the concept, under a specific context, either positively or negatively.
\end{itemize}

The reason for using a concept as a starting point is that concepts are, as was seen in the previous sub-section, the basic elements of legal research. Nevertheless, for interpretability, it is important that the user can easily assess the relevancy of the concept and context and we have therefore opted for translating the concept, legislation and context into a question.

%As we will see later in the paper, our questions were generated automatically by using ChatGPT. While manually created questions are not excluded, we believe that automated questions creation is more scalable and erroneous questions can be easily filtered out by the user. We discuss all these points in more depth later.

Once questions are being identified, our goal is to automatically extract from judgement as many paragraphs as possible which can serve as either a positive or a negative example for the concept and under the context defined in the question.

\section{Related work}
\label{sec:sota}

Following the analysis given in \cite{martinez2023survey} and the recent developments in neural QAS, 
it is safe to say that Legal QAS fall into one of three categories:
\begin{enumerate}
    \item knowledge-based systems (using ontologies and linked data), e.g. \cite{fawei2018methodology, filtz2021linked},
    \item retrieve, re-rank and interpret systems, e.g. \cite{zhong2020building},
    \item large langue model based systems, e.g. \cite{chung2022scaling}.
\end{enumerate}

An example of the first approach is given in \cite{fawei2018methodology}. The authors manually build an ontology
for criminal law and supplement it with rules. Answering questions is possible thanks
to an inference engine, which draws conclusions based on the interpretations of the legal concepts and 
application of the rules. The key benefits of that approach are explainability and accuracy, but this approach
suffers from scalability issues, since the ontology and logical rules are manually created. Moreover 
the inference engines are slow, which negatively influences the performance of such systems \cite{martinez2023survey}.

The example of the second approach is given in \cite{zhong2020building}, where the authors describe a system
helping engineers in China to find legislation regulating the design and development of buildings.
The system is based on a combination of sparse retrieval and deep learning. When the user asks a question,
the system searches for relevant law in a database containing the relevant legislation using TF-IDF term
weighting scheme \cite{manning1999foundations}. Then for top-n returned passages (the system does not have a re-ranking
module) a Chinese BERT-like model is applied. The model is fine-tuned on a collection of question-passage-answer triplets, to 
perform extractive question answering (the answer has to be present literally in the passage). The system is interpretable
on the human level (the returned passages and the answer location might be inspected manually), but has a limited scalability,
In addition, the terms in the question must match the terms in the legislation directly, due to the application of 
a sparse retrieval algorithm. To overcome that problem a thesaurus would have to be built or a dense retriever such as DPR \cite{karpukhin2020dense} would have
to be applied.

ChatGPT \cite{chung2022scaling} is an example of a system of the third type. Even though these system achieve 
state-of-the-art results in legal QA\footnote{\cite{openai2023gpt4} claims that GPT 4.0 scores among 10\% of the top test takers 
for the Uniform Bar Exam, but the paper is a pre-print and ,,the authors'' do not provide the most important technical details
of the system}, they have one very important limitation -- they are not explainable and they have a 
tendency to hallucinate, i.e. they can make up facts. In the USA, two lawyers were recently fined for citing non-existent cases
reported by ChatGPT \cite{ap2023lawyers}.

Our system is most similar to the second approach, but it differs in the following ways:
\begin{enumerate}
    \item the system does not give an answer, but finds positive and negative examples,
    \item the system takes a legal concept as the input,
    \item the retrieval is based solely on a deep neural network,
    \item the queries (questions) are generated automatically by a large language model.
\end{enumerate}

We claim that the system presented in the next two sections achieves a similar accuracy level to the ones mentioned above while being more explainable and scalable.

%Overview sota with regards to the three paramters:
%
%Explainability is obtained by keeping the original text and keeping references to original judgments
%Accuracy is obtained by not answering a question but framing the answer using real positive and negative examples
%Scalabilty is obtained by the use of unsupervised learning.

%\cite{zhong2020building} is an example of a system that utilizes deep learning for factoid
%question answering, following the retrieve-extract paradigm. The authors developed a system
%that is aimed at answering questions related to the Chinese building regulations. 
%The corpus contains the documents from the Chinese law related to buildings. These documents
%are split into passages following the structure of provisions -- each provision is treated 
%as a tree and for each non-leaf node a passage containing the content of parent nodes and all 
%descendant nodes is created.
%
%The retrieval model is based on a sparse representation of the passages -- the authors use TF-IDF 
%\cite{manning1999foundations} vector model for retrieval. 
%The training size of the dataset contains 2500 triplets, while the validation and the testing
%part contain 500 triplets each.
%
%The authors report very high accuracy of the answers (95\% exact match score for factoid questions 
%and 90\% for definitional questions), but this only comprises the answer extraction module. 
%The authors do not provide a similar scores for the complete system, but only report the user 
%evaluation in terms of easy of use, quality of answers and time spent by the user. The overall performance 
%of the system judged by the users is better compared to Baidu search engine and regulation document 
%database (however the last system is judged better in terms of accuracy of the answers).



\section{Methodology}
\label{sec:method}

As it was discussed in Section \ref{sec:sota} there are two dominant approaches 
based on neural networks that are used to solve the problem of QA:
\begin{enumerate}
  \item retrieve and extract/generate answer, e.g. \cite{}, %RAG
  \item generate an answer using a large language model (LLM) \cite{}. %GPT-3
\end{enumerate}

Our approach -- LCES -- differs from them, since we:
\begin{enumerate}
  \item we do not require that the user poses a question to the system, since the system might generate a question for the user,
    taking into account the legislation,
  \item we do not answer the question, but give positive and negative examples, taken directly from the relevant court decisions.
\end{enumerate}
Our approach is most similar to the retrieve and generate paradigm, since when searching for relevant information we use
a cross-encoder (typically used to re-score the results of a sparse or a dense retriever). Yet we also introduce an important
difference in that respect, by taking individual sentences as the primary means for bearing atomic pieces of information.  
As a result we resolve the problem of hallucination present in the second mentioned approach, i.e. the approach based on LLMs.

The procedure of finding the relevant examples is divided into the following steps:
\begin{enumerate}
  \item Given a legal concept, generate a question that asks if the concept was applicable in a given context,
    using a first-order language\footnote{The first-order language is used in the Tarski's sense -- to differentiate
    between the first-order language and the meta-language that is used to speak about the first-order language.}, e.g. 
    \textit{...} % example
  \item Having the question generated, we use a binary classification model to find sentences in the court cases that answer the question. 
    We use the predicted probability not only to sort these sentences, but also to filter sentences below a defined threshold.
    As a result this step might generate an empty set, meaning that in our corpus there are no relevant court decisions. 
    This prevents the system from presenting passages that are irrelevant, but also limits the possibility of false
    classification of the sentences.
  \item In the last step we extend the sentence to include the preceding and the following sentence forming a three-sentence passage
    and ask an instruction-following model to judge if the example gives a positive or a negative answer to the generated question.
    We use that prediction to classify the passage as as positive or a negative example.
\end{enumerate}

The detailed description of the steps are given in the following sections.

\subsection{Question generation}


\subsection{Sentence retrieval}


\subsection{Passage classification}

\section{Experiments}
\label{sec:experiments}

The aim of the experiment in this section is to give a minimal validation of the methodology presented in the previous section. We are currently planning a more substantial experiment which will take place as part of a graduate course.

In Sec. \ref{sec:method}, we have described the generation of 55 questions (cf. Table \ref{tab:question-examples}). We now proceed with the description of the experiment.
For training the sentence retrieval module we have used SQuAD v.2 \cite{rajpurkar2018know} sentence-split by the Stanza 
library \cite{qi2020stanza}. 
The procedure gave more than 650 thousand training examples and more than 63 thousand validation examples. There were 87 thousand
positive examples (see Sec. \ref{sec:method}) in the training set (13\%) and almost 6 thousand positive examples in the validation set (9.5\%).
We have fine-tuned two models on the data: BERT large uncased \cite{devlin2018bert} (\texttt{bert-large}) and 
Albert XXL v.1 \cite{lan2019albert} (\texttt{albert-xxl}). The results of the training are given in Table \ref{tab:bert-albert}.

\begin{table}[htbp]
    \centering\begin{tabular}{l r r r r}
        \hline
        \textbf{Model} & \textbf{Accuracy} & \textbf{F1-score} & \textbf{Steps} & \textbf{Time\footnote{Using 4 instances of A100 40GB NVIDIA GPU}} \\
        \hline
        \texttt{bert-large} & 95.77 & 78.15 & 50 000 & 20h \\
        \hline
        \texttt{albert-xxl} & \textbf{97.05} & \textbf{84.14} & 5 000 & 24h \\
        \hline
    \end{tabular}
    \caption{The results of the training BERT large uncased and Albert XXL on the sentence-split SQuAD dataset.}
    \label{tab:bert-albert}
\end{table}

Validating the models on the testing data, we have found that \texttt{bert-large} gave results much worse than \texttt{albert-xxl}, even though it was trained 10-times longer 
(in terms of the number of steps, not the wall clock time; the total training time for Albert is 4h longer). 
As a result we have decided to use the fine-tuned \texttt{albert-xxl} 
in the following experiment. 

As a source for GDPR related court case decisions, we have decided to use data from GDPRHub\footnote{\url{https://gdprhub.eu}} -- a \url{noyb.eu} financed site supported by
a large group of volunteers, who standarize and format European courts' decisions. Besides the text of the decision (automatically translated into English) the database
contains user-created description which includes sections such as: facts, holding, comment and further resources.

Using GDPRHub ability to filter cases according to cited articles, we have downloaded all the decisions available for articles 44-46 (81 decisions in total, 47 unique decisions) and extracted the facts and the holdings of their English 
summary. These sections were sentence-split using the Stanza library \cite{qi2020stanza}, yielding 1201 sentences
of which 1140 were unique. 

We have then applied our model for automatically ranking pairs of a question (among the 55 generated by ChatGPT) and sentences in the processed court cases.
For each question, we have manually inspected the top 5 results
produced by the classification model, even for cases when the model yielded probability below 0.5, in order to see if the default
decision threshold works as expected. As a result we were able to identify 0.65 as the optimal (equal error rate) threshold for our 
experiment. There were 18 such sentences (with 2 duplicates, obtained for similar questions) with 4 false positives (77.78\% precision).
There were also 4 false negatives (identified among the top 5 results below 0.65 threshold), giving 77.78\% recall and the same F1-score. 
The results obtained by our model are on par with the state-of-the-art neural passage retrieval model -- DPR \cite{karpukhin2020dense},
which obtains 72.24 accuracy on the top 5 results tested on the Natural Questions dataset \cite{kwiatkowski2019natural}. Obviously, 
our results should be treated with a grain of salt, since we were biased towards positive assessment of our approach and the 
dataset is orders of magnitude smaller, yet they still show that the approach produces meaningful preliminary results.

For the last step, we have provided all results with probability above 0.65 and the 4 remaining positive results below that score
to the Flan-T5-large model \cite{wei2021finetuned,raffel2020exploring}, to obtain an answer if the example is positive or negative
with respect to the given question. The sentences were extended with one preceding and one following sentence, to form a richer context.
We have manually reviewed the answers provided by the model. Out of 32 analyzed examples (including those  
that were previously judged invalid), 18 (\textbf{56.25\%}) were correct (this result included 2 duplicates) and 14 were incorrect (this set
also included 2 duplicates). The obtained result is very similar to the state-of-the-art QA system -- Dense Passage Retriever,
which -- depending on the employed dataset -- for the end-to-end evaluation yields 35.8 -- 57.9 accuracy. We claim that this 
supports our claim that the presented approach yields accurate results to the level achievable by the best QA systems 
available currently, but at the same time, enjoys performance and interpretability, which allows users to easily filter out false results.

The following example shows the result for the concept: \textit{Legally binding and enforceable instrument}. The entire data and results described in this paper can be found on Github\footnote{\url{https://github.com/apohllo/jurix-2023}}.
Among others, ChatGTP generated a following question for that concept: \textit{Does the instrument in
question meet the criteria for being legally binding under relevant laws?}
In the GDPRHub corpus the \texttt{albert-xxl} model found a sentence, which after an expansion formed
the following example: \textit{In its privacy policy the controller simply
informed consumers about data transfers to third parties and countries, without
any further legal effect.  \textbf{This document did not constitute a legally binding
contract offered to customers by the controller, as the Consumer Center
suggested.} The court also held that the Consumer Center's claim with regard to
the cookie banners was unfounded.} In the last step, the Flan T5 model judged that the
answer is \textit{no}, so this context would be presented to the user as a
negative example.


\section{Conclusion}

In this paper we introduced a sub-system of QAS and discussed its merits and a possible methodology for its creation. We also demonstrated via a small experiment how the methodology can be implemented and what kind of results it might return.

The work introduced in this paper is an initial work towards high-quality CESs creation. As such, there are numerous ways to optimize the process, which we consider as future work. Due to lack of space, we mention but a few of them next.

In the paper we have used a subjective and biased method in order to evaluate the relationship between the score provided by our model and the actual relevance of examples to the questions. We plan on conducting an evaluation with the help of law students.

Another form of improvement would be the fine-tuning of pre-trained models with legal datasets for the purpose of sentence extraction.

We also plan on generating better and more numerous questions for each concept. Either via improving the prompt, using LLMs fine-tuned specifically on legal data or by using an entirely different approach.

Similarly, the selection of the relevant context for each answer (currently fixed on the preceding and following sentences) can be much improved.

Lastly, we would like to apply the method on a much larger set of relevant court case judgements. For example by using the one provided by \url{case.law}.



\bibliographystyle{vancouver}
\bibliography{main}

%\bibitem{filtz21}
%Filtz, E. et al. "The linked legal data landscape: linking legal data across different countries." \textit{Artificial Intelligence and Law 29.4} (2021): 485-539.
%
%\bibitem{kelly21}
%Sanderson, J. et al. A Practical Guide to Legal Research. \textit{Lawbook Co.}, (2021).
%
%\bibitem{martinez23}
%Martinez-Gil, J. "A survey on legal question–answering systems." \textit{Computer Science Review 48} (2023): 100552.
%
%\bibitem{wyner18}
%Fawei, B. et al. A Methodology for a Criminal Law and Procedure Ontology for Legal Question Answering. \textit{ Semantic Technology. JIST 2018. Lecture Notes in Computer Science}, vol 11341. Springer, Cham. (2018)

\end{document}
