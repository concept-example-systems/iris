Main source: A practical Guide to Legal Research - Kelly and Sanderson

Steps of legal research (which is an iterative process:
1. identification of the relevant facts;
2. identification of the legal issues;
3. identification and interpretation of the rules that govern the legal issues;
4. application of the rules to the facts; and
5. conclusion/s.

Alban and Salim case

Step 1:

Are Aban and Salim refugees? What does the term refugee mean?
Is persecution relevant to refugee status? If so, how?
Can Aban and Salim be persecuted because they are homosexual?
Does the law of Australia or Pakistan apply?
Is there relevant international law?

Distilling the facts: Refugee; Persecution; Homosexual

Step 2:

Requires (among 7 steps)
• the early identification of relevant search terms;
• an understanding of search syntax (how to employ your search terms);

Step 4:

Legal reasoning
Most often reasoning by analogy: where we treat like cases alike, in that we look to facts, issues, precedent, principles and policy adopted in a case to see if we can predict the outcome of the issue confronting us, relative to earlier precedent. Thus, using reasoning by analogy we might compare differences and distinguish a case.

Also inductive and deductive reasonings are used

Step 5: conclusion

The main goal of step 2 is to produce search terms. Together with their synonyms, they serve for the first iteration of legal research.

An additional step is to define the search terms using secondary sources such as dictionaries and possibly obtain further keywords.

For the example, the additional term "well-founded fear of persecution" can be obtained

Build a search term, such as [“well-founded fear of persecution” and (gay or homosex!)]

Step 3: legislation
We then find the most relevant legislation and most recent version that deals with our legal issue. We also look for referenced legislation, etc. Normally, legislation is accessible and searchable via public databases. In addition, the best way to find the relevant legislation is via secondary sources like legal encyclopedia.

Step 3: case law
In a common law system, case law is central to
understanding, interpreting and applying the law.

From the legislation, we have questions such as:
- what is a well-founded fear of persecution?
- is homosexuality a ‘social group’ for the purposes of the Migration Act 1958?

In order to resolve these questions, we first try to figure out the information from the legislation and if not possible, we look at case law. "Case law should assist us to determine the meaning of ‘well-founded fear’ and ‘social group’" (cited from book).