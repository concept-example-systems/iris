\subsection{The legal research process}

Before we can present Concept-Example Systems, we would like to understand why they might be useful. We therefore devote the first section to an exposition of the legal research process. For that purpose, we follow the description by Sanderson and Kelly in their ``A practical Guide to Legal Research'' \cite{sanderson2021practical} and take advantage of their running example. 
%It should be noted that while the example takes place in Australia and is based on Australian law, the process in other legal systems does not differ much when considering legal research.
The example is about two Pakistani men, Aban and Salim, who have arrived in Australia with visitor visas and sought refugee status before their visas have expired. They claim they would be prosecuted for being homosexual if they are forced to return to Pakistan and risk being beaten or killed.

Our goal is to build their case for an asylum status in Australia and for that, we need to conduct legal research. 
In their guide, Sanderson and Kelly define the following steps of legal research:

\begin{definition}[Steps of Legal Research \cite{sanderson2021practical}]
The main steps of legal research are:
\begin{enumerate}
    \item identification of the relevant facts;
    \item identification of the legal issues;
    \item identification and interpretation of the rules that govern the legal issues;
    \item application of the rules to the facts; and
    \item conclusion/s.
\end{enumerate}
\end{definition}

Our focus in this paper is step 3. We therefore briefly conclude the findings of the first two steps and ignore steps 4 and 5.

The first step concludes with the following facts - Refugee; Persecution; Homosexual. 
For identifying the legal issues (Step 2), we proceed to identify key legal questions with regards to the facts.

\begin{itemize}
\item Are Aban and Salim refugees? What does the term refugee mean?
\item Can Aban and Salim be persecuted because they are homosexual?
\end{itemize}

Steps 3 guides us in the legal research, which is the main topic of this paper. The key questions to answer in this step are:

\begin{enumerate}
\item What is the relevant legislation? 
\item What cases, if any, have considered similar issues? 
\item Are there relevant international treaties or conventions?
\end{enumerate}

Our paper focuses on legal research about cases, i.e. item 2 above. We therefore omit items 1 and 3. We nevertheless, follow their reasoning and conclude that the relevant legislation is Migration Act 1958 (Cth)\footnote{\url{http://www5.austlii.edu.au/au/legis/cth/consol_act/ma1958118/index.html}}, which references the United Nations Convention Relating to the Status of Refugees\footnote{\url{https://www.unhcr.org/media/convention-and-protocol-relating-status-refugees}}.

From the legislation, we have questions such as:
\begin{itemize}
\item what is a well-founded fear of persecution?
\item is homosexuality a ‘social group’ for the purposes of the Migration Act 1958?
\end{itemize}

The guide states that ``Case law should assist us to determine the meaning of ‘well-founded fear’ and ‘social group'' \cite{sanderson2021practical} and suggests three methods for doing that: (1) by using search terms, (2) by starting from a known similar case, (3) by looking on judicial analysis of a certain legislation. In this paper we focus on the first option.
Modern case law search is mainly done by using special tools, such as the ones offered by LexisNexis and Thomson Reuters Westlaw, or by publicly available indexes, such as EUR-Lex\footnote{\url{http://eur-lex.europa.eu}}.

The main search process includes using search terms and filters in order to locate relevant judgements. If we input in an Australian case law search tool the keywords ``well-founded fear'', together with keywords such as ``homosexuality'' and ``prosecution'', many relevant cases are returned, such as ``S395 v Minister for Immigration and Multicultural Affairs''. 

Once a set of relevant cases was identified, the legal professional needs to read the cases, as summaries or digests are not enough.
A reading of the above mentioned case gives the following paragraph: ``the Tribunal had misdirected itself on the issue of discretion and had not properly considered the appellants’ claims that they had a real and well-founded fear of persecution if they returned to Bangladesh. The real question was whether the harm that would be suffered was such that, {\bf by reason of its intensity or duration, the person cannot reasonably be expected to tolerate it}''.

This paragraph and its concluding remark gives a positive example for a high court decision whether the keyword ``well-founded fear'' holds under the context ``homosexuality'' and ``prosecution'' and under the Australian ``Migration Act 1958''. By using this example, the legal professional can return to her case at hand at determine if the example is similar or not. Clearly, further examples and especially a mix of positive and negative ones would provide the most useful help in determining that.

We argue in this paper that one of the most useful questions for a legal QAS would be to show positive and negative examples of a concept under a certain context. When considering a general purpose legal QAS, it would need to understand the meaning of the question and would not be optimized for finding examples, thus increasing the chance for error, as well as hampering interpretability. It should be noted though, that there are also many cases when the most useful question to ask is not of the above form.

We can now describe a new type of QASs which is tailored for the task above. A methodology for creating such a system, as well as a simple experiment, are described in sections \ref{sec:method} and \ref{sec:experiments}.

\subsection{Concept-Example Systems (CESs)}

A Concept-Example System (CES) differs from a QAS in two aspects:

\begin{itemize}
 \item The starting point for generating a question-answer pair is a legal concept and not a question.
 \item We do not look for specific answers but for as many extracts from court judgements which resolve the concept, under a specific context, either positively or negatively.
\end{itemize}

The reason for using a concept as a starting point is that concepts are, as was seen in the previous sub-section, the basic elements of legal research. Nevertheless, for interpretability, it is important that the user can easily assess the relevancy of the concept and context and we have therefore opted for translating the concept, legislation and context into a question.

%As we will see later in the paper, our questions were generated automatically by using ChatGPT. While manually created questions are not excluded, we believe that automated questions creation is more scalable and erroneous questions can be easily filtered out by the user. We discuss all these points in more depth later.

Once questions are being identified, our goal is to automatically extract from judgement as many paragraphs as possible which can serve as either a positive or a negative example for the concept and under the context defined in the question.

\section{Related work}
\label{sec:sota}

Following the analysis given in \cite{martinez2023survey} and the recent developments in neural QAS, 
it is safe to say that Legal QAS fall into one of three categories:
\begin{enumerate}
    \item knowledge-based systems (using ontologies and linked data), e.g. \cite{fawei2018methodology, filtz2021linked},
    \item retrieve, re-rank and interpret systems, e.g. \cite{zhong2020building},
    \item large langue model based systems, e.g. \cite{chung2022scaling}.
\end{enumerate}

An example of the first approach is given in \cite{fawei2018methodology}. The authors manually build an ontology
for criminal law and supplement it with rules. Answering questions is possible thanks
to an inference engine, which draws conclusions based on the interpretations of the legal concepts and 
application of the rules. The key benefits of that approach are explainability and accuracy, but this approach
suffers from scalability issues, since the ontology and logical rules are manually created. Moreover 
the inference engines are slow, which negatively influences the performance of such systems \cite{martinez2023survey}.

The example of the second approach is given in \cite{zhong2020building}, where the authors describe a system
helping engineers in China to find legislation regulating the design and development of buildings.
The system is based on a combination of sparse retrieval and deep learning. When the user asks a question,
the system searches for relevant law in a database containing the relevant legislation using TF-IDF term
weighting scheme \cite{manning1999foundations}. Then for top-n returned passages (the system does not have a re-ranking
module) a Chinese BERT-like model is applied. The model is fine-tuned on a collection of question-passage-answer triplets, to 
perform extractive question answering (the answer has to be present literally in the passage). The system is interpretable
on the human level (the returned passages and the answer location might be inspected manually), but has a limited scalability,
In addition, the terms in the question must match the terms in the legislation directly, due to the application of 
a sparse retrieval algorithm. To overcome that problem a thesaurus would have to be built or a dense retriever such as DPR \cite{karpukhin2020dense} would have
to be applied.

ChatGPT \cite{chung2022scaling} is an example of a system of the third type. Even though these system achieve 
state-of-the-art results in legal QA\footnote{\cite{openai2023gpt4} claims that GPT 4.0 scores among 10\% of the top test takers 
for the Uniform Bar Exam, but the paper is a pre-print and ,,the authors'' do not provide the most important technical details
of the system}, they have one very important limitation -- they are not explainable and they have a 
tendency to hallucinate, i.e. they can make up facts. In the USA, two lawyers were recently fined for citing non-existent cases
reported by ChatGPT \cite{ap2023lawyers}.

Our system is most similar to the second approach, but it differs in the following ways:
\begin{enumerate}
    \item the system does not give an answer, but finds positive and negative examples,
    \item the system takes a legal concept as the input,
    \item the retrieval is based solely on a deep neural network,
    \item the queries (questions) are generated automatically by a large language model.
\end{enumerate}

We claim that the system presented in the next two sections achieves a similar accuracy level to the ones mentioned above while being more explainable and scalable.

%Overview sota with regards to the three paramters:
%
%Explainability is obtained by keeping the original text and keeping references to original judgments
%Accuracy is obtained by not answering a question but framing the answer using real positive and negative examples
%Scalabilty is obtained by the use of unsupervised learning.

%\cite{zhong2020building} is an example of a system that utilizes deep learning for factoid
%question answering, following the retrieve-extract paradigm. The authors developed a system
%that is aimed at answering questions related to the Chinese building regulations. 
%The corpus contains the documents from the Chinese law related to buildings. These documents
%are split into passages following the structure of provisions -- each provision is treated 
%as a tree and for each non-leaf node a passage containing the content of parent nodes and all 
%descendant nodes is created.
%
%The retrieval model is based on a sparse representation of the passages -- the authors use TF-IDF 
%\cite{manning1999foundations} vector model for retrieval. 
%The training size of the dataset contains 2500 triplets, while the validation and the testing
%part contain 500 triplets each.
%
%The authors report very high accuracy of the answers (95\% exact match score for factoid questions 
%and 90\% for definitional questions), but this only comprises the answer extraction module. 
%The authors do not provide a similar scores for the complete system, but only report the user 
%evaluation in terms of easy of use, quality of answers and time spent by the user. The overall performance 
%of the system judged by the users is better compared to Baidu search engine and regulation document 
%database (however the last system is judged better in terms of accuracy of the answers).
