As it was discussed in Section \ref{sec:sota} there are two dominant approaches 
based on neural networks that are used to solve the problem of QA:
\begin{enumerate}
  \item retrieve and extract/generate answer, e.g. \cite{lewis2020retrieval}, %RAG
  \item generate an answer using a large language model (LLM), e.g. \cite{brown2020language}. %GPT-3
\end{enumerate}
Our approach -- LCES -- differs from them, since:
\begin{enumerate}
  \item we do not require that the user poses a question to the system, since the system might generate a question for the user,
    taking into account the legislation,
  \item we do not answer the question, but give positive and negative examples, taken directly from the relevant court decisions.
\end{enumerate}
Our approach is most similar to the retrieve and generate paradigm, since when searching for relevant information we use
a cross-encoder (typically used to re-score the results of a sparse or a dense retriever). Yet we also introduce an important
difference in that respect, by taking individual sentences as the primary means for bearing atomic pieces of information.  
As a result we resolve the problem of hallucination present in the second mentioned approach, i.e. the approach based on LLMs.

The procedure of finding the relevant examples is divided into the following steps:
\begin{enumerate}
  \item Given a legal concept, generate a question that asks if the concept was applicable in a given context,
    using a first-order language\footnote{The first-order language is used in the Tarski's sense -- to differentiate
    between the first-order language and the meta-language that is used to speak about the first-order language. In this 
    context it means that the questions contains the legal concept as a normal lexical unit, rather than a quoted phrase.
    For example, for the concept \textit{data subject rights} a question in first-order language would be 
      ,,Were the data subjects informed about their rights regarding their personal data?'', vs. 
      a meta-language level question ,,Does the passage contain a $\ll$data subject rights$\gg$ phrase?''.}.
    For example, the term \textit{legally binding and enforceable instrument} generates the following question:
    \textit{Does the instrument in question meet the criteria for being legally binding under relevant laws?}
  \item Having the question generated, we use a binary classification model to find sentences in the court cases that answer the question. 
    We use the predicted probability not only to sort these sentences, but also to filter sentences below a defined threshold.
    As a result this step might generate an empty set, meaning that in our corpus there are no relevant court decisions. 
    This prevents the system from presenting passages that are irrelevant, but also limits the possibility of false
    classification of the sentences.
  \item In the last step we extend the sentence to include the preceding and the following sentence forming a three-sentence passage
    and ask an instruction-following model to judge if the example gives a positive or a negative answer to the generated question.
    We use that prediction to classify the passage as a positive or a negative example.
\end{enumerate}

It has to be stressed that none of the steps uses a domain-specific (legal) question-answering dataset. We only use an open-domain
dataset (SQuAD) and general generative models (ChatGPT and Flan-T5). This is a very important feature of our approach,
since the oder neural systems related to legal question answering are typically trained on legal datasets.
The detailed description of the procedure is given in the following sections.

\subsection{Question generation}

\textit{Tomer: provide the rationale for the concept selection and the prompt for ChatGPT used to generate the questions.}

We use ChatGPT to generate the questions using the legal concepts as the input. For each legal concept we have generated
4 specific and 1 general question. In general the number of questions is a parameter of the approach and a higher number
should yield a better recall of the results. So far we have left the influence of that parameter on the result as well
as a selection of particular language model for future research.

Since we test our approach on GDPR, we assume that ChatGPT has a good knowledge
of the relevant legislation, since it should be present in the corpus, the
system was trained on. As the results show (cf. Table
\ref{tab:question-examples}), this is a valid assumption, however for a solution
that is independent from the knowledge of the language model (i.e. taking into
account some amendments that were applied recently), we could provide the
relevant provision as a part of the prompt.

\begin{table}[htbp]
  \centering\begin{tabular}{l p{6cm}}
    \hline
    \textbf{Legal concept} & \textbf{Generated question} \\
    \hline
Enforceable data subject rights & Were the data subjects informed about their rights regarding their personal data?\\
\hline
Effective legal remedies for data subjects & Are there established procedures within the organization for data subjects to seek legal remedies?\\
\hline
Legally binding and enforceable instrument & Does the instrument in question meet the criteria for being legally binding under relevant laws?\\
\hline
Binding corporate rules & Does the organization have binding corporate rules in place for data protection?\\
\hline
Standard data protection clauses & Has the organization implemented Standard Contractual Clauses (SCCs) for data transfers?\\
\hline
Approved code of conduct & Has the organization's code of conduct been officially approved by the relevant authority?\\
\hline
Approved certification mechanism & Has the organization's certification mechanism been officially approved by the relevant authority?\\
\hline
Competent supervisory authority & Has the organization received approvals or guidance from the competent supervisory authority for its data protection practices?\\
\hline
Contractual clauses & Does the organization incorporate specific data protection clauses in its contracts with third parties?\\
\hline
Administrative arrangements & Does the organization have administrative arrangements in place that include provisions for data subject rights?\\
\hline
Consistency mechanism & Does the organization employ a consistency mechanism to ensure uniform data protection practices?\\
\hline

  \end{tabular}
  \caption{Example questions generated by ChatGPT for the legal concepts taken from GDPR.}
  \label{tab:question-examples}
\end{table}

We argue that this application for generative AI in the legal domain is justified and interpretable. 
Generative AI is very good at rephrasing sentences and expanding concepts (as the results show clearly). 
By applying this method we remove the burden of generating similar question from the user, 
so the proposed approach preserves time of the user. On the other hand, since the generated questions 
might be presented to the user, the process is interpretable at the human level \cite{martinez2023survey}, meaning that
even though the workings of the neural network cannot be tracked, its outcome can be understood by humans and 
possible errors might be identified as wrongly generated questions. Once again removal of such invalid questions 
is a much faster process than their invention by the potential users of the solution.

\subsection{Sentence retrieval}

The next step in the procedure is the retrieval of the relevant pieces of judgments that include answers to the
generated questions. As explained in the Section \ref{sec:experiments}, since the collection of text we use in our experiments
is small, compared to other QA dataset, we have decided to use a simplified approach for retrieving the relevant
passages of the decisions. Instead of first retrieving a subset of the data using a sparse or a dense retriever,
we have implemented only a cross-encoder, which is typically used for re-scoring the results on a subset of documents.
In our approach the cross-encoder is run for each generated question and each available passage. 

As the cross-encoder we have used a binary classification model by fine-tuning AlBERT \cite{lan2019albert} model (from the BERT family
\cite{devlin2018bert} of encoder-only transformer models). We have used SQuAD dataset \cite{rajpurkar2016squad},
specifically the second version with unanswerable questions \cite{rajpurkar2018know}. We have implemented
an indirect method of contrastive learning (i.e. the contrastive goal is not encoded in the loss but in the preparation of
the  training examples). A typical procedure for SQuAD 2.0, to build a binary classifier judging if the passage 
contains an answer to the question would be to use the unanswerable questions together with the passages as the 
negative examples and the answerable questions with their passages as the positive examples. Yet, that would yield
a highly imbalanced dataset, which does not reflect the distribution of positive and negative question -- passage 
pairs of the application domain (in SQuAD 2.0 there are 20\% of unanswerable pairs, while 
for the retrieval more than 99\% of passages are irrelevant). 

To resolve that problem we have pre-processed SQuAD 2.0 by splitting each passage into individual sentences and identified
those sentences that included the whole answer. This is straightforward operation, since SQuAD implements the extractive
QA paradigm which requires that the answer is directly present in the passage as a sequence of consecutive tokens.
The sentence that includes those tokens together with the question is taken as a positive example, while
all the other sentences \textbf{from the same passage} and that question are treated as the negative examples.
We call this approach an indirect contrastive learning, since the negative examples will include information 
which is highly relevant for the question, but they do not include the actual answer. For the unanswerable questions
we also split the passage into individual sentences and treat all of them as negative examples. As a result we 
obtain a dataset that includes a high number of negative examples, which are still highly relevant with respect to the 
question.

Since the model outputs a probability score for the question -- sentence pair, it is possible to sort the sentences according
to the decreasing probability of the sentence containing the answer to the question. But since this is a binary classification
model, by default paris with probability below 0.5 will be judged as not containing the answer. This threshold might be increased
to obtain a higher accuracy of results. 

Once again the workings of this module are interpretable at the human level for the sentences that were classified as 
containing the answer, since they might be presented directly to the user and the user might check which question
yield a positive result for the sentence. Such sentences might be then ignored by the user. However for the negative 
results there is no way to check if the system has not omitted some relevant results. Yet from practical point of view
-- assuming the system will produce considerable amount of examples in the next step -- such an 
omission would not be a problem, since the goal is to provide positive and negative examples of the interpretation of the
legal concept, rather than retrieve \textbf{all} usages of that concept in the decisions.

\subsection{Passage classification}

As the last stage of processing we classify the sentences as positive or negative examples. Since in the second step
we process individual sentences, we first expand the context by taking one preceding and one following sentence, in 
order to provide better understanding for the model which will make the decision. To classify the such extended 
context into one of the classes we use an instruction following model, in that case Flan-T5 \cite{chung2022scaling}. 
This is a T5-based model \cite{raffel2020exploring}, fine-tuned to follow instructions on a large group of various
English datasets. The prompt sent to the model is given in Figure \ref{fig:flan-instruction}.

\begin{figure}[htbp]
  \texttt{Given the context: '<context>' answer the following question by reasoning step-by-step: <question>} 
  \caption{The prompt used to ask Flan-T5 for classification of the examples.}
  \label{fig:flan-instruction}
\end{figure}

The result of the prompt is an explanation of why the model believes that the answer is either positive or negative.
It allows to group the examples into positive and negative by checking if the answer contains `yes'. The additional
outcome is availability of a justification for the answer the model has given. Once again, this makes the model
interpretable on the human level, since even if the provided judgment is invalid, since the model do not perform 
a real reasoning, the user is able to assess if the justification is sensible. 

We haven't used ChatGPT in the last step (as in the first step), since the number of queries to the API would incur
a high cost. Flan-T5 is not as good as ChatGPT, but it's performance is one of the best \cite{chia2023instructeval} 
among the open-source models that demonstrate the instruction following capability.