\cite{zhong2020building} is an example of a system that utilizes deep learning for factoid
question answering, following the retrieve-extract paradigm. The authors developed a system
that is aimed at answering questions related to the Chinese building regulations. 
The corpus contains the documents from the Chinese law related to buildings. These documents
are split into passages following the structure of provisions -- each provision is treated 
as a tree and for each non-leaf node a passage containing the content of parent nodes and all 
descendant nodes is created.

The retrieval model is based on a sparse representation of the passages -- the authors use TF-IDF 
\cite{manning1999foundations} vector model for retrieval. Then for top-n returned passages the
apply a Chinese BERT-like model that was trained on the Chinese Wikipedia and the content of the 
regulations. The model is fine-tuned on a collection of question-passage-answer triplets, to 
perform extractive question answering (the answer has to be present literally in the passage).
The training size of the dataset contains 2500 triplets, while the validation and the testing
part contain 500 triplets each.

The authors report very high accuracy of the answers (95\% exact match score for factoid questions 
and 90\% for definitional questions), but this only comprises the answer extraction module. 
The authors do not provide a similar scores for the complete system, but only report the user 
evaluation in terms of easy of use, quality of answers and time spent by the user. The overall performance 
of the system judged by the users is better compared to Baidu search engine and regulation document 
database (however the last system is judged better in terms of accuracy of the answers).